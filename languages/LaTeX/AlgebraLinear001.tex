\documentclass[a4paper, 12pt]{article}
\usepackage[top=2cm, bottom=2cm, left=2.5cm, right=2.5cm]{geometry}
\usepackage[utf8]{inputenc}
\usepackage{amsmath, amsfonts, amssymb}
\usepackage{multicol}
\begin{document}

\begin{center}
\textbf{Álgebra Linear - Provas Matemáticas \\ Aluno: Eduardo Santos de Castro}
\end{center}

\begin{enumerate}
\item \textbf{Operações com Escalares}

\textbf{$\alpha(\beta A) = (\alpha \beta) A$ (Associativa)}

Por definição de produto por escalar, multiplicamos um valor $\alpha$ por cada elemento $a_{ij}$ da matriz A. Considerando que $\alpha$ e todos esses elementos são números reais, tem-se que:

$$
[\alpha(\beta A)]_{ij} 
= \alpha [\beta A]_{ij}
= (\alpha \beta)[A]_{ij}
= [(\alpha \beta A)]_{ij}
= \alpha (\beta A) = (\alpha \beta) A.
$$ \\


\textbf{$\alpha (A B) = (\alpha A) B = A (\alpha B)$ (Associativa)} 

Sejam $
A=[a_{ij}]_{m \times p}, 
B=[b_{ij}]_{p \times n}, 
(\alpha A)B=[c_{ij}]_{m \times n}, 
A (\alpha B)=[d_{ij}]_{m \times n}, 
\alpha (AB)=[e_{ij}]_{m \times n}$ 
e $\alpha \in \mathbb{R}$

$$
[c_{ij}] 
= \sum\limits_{k=1}^{n}(\alpha \cdot a_{ik}) \cdot b_{kj}
= \alpha \cdot \sum\limits_{k=1}^{n} a_{ik} \cdot b_{kj}
= [e_{ij}]
$$ \

$$
[d_{ij}] 
= \sum\limits_{k=1}^{n}a_{ik} \cdot(\alpha \cdot b_{kj})
= \alpha \cdot \sum\limits_{k=1}^{n} a_{ik} \cdot b_{kj}
= [e_{ij}]
$$

então, isso implica dizer que $\alpha (A B) = (\alpha A) B = A (\alpha B)$. \\ \\


\textbf{$\alpha (A \pm B) = \alpha A \pm \alpha B$ (Distributiva)}

Sejam $[A]_ij, [B]_ij$ e $\alpha \in \mathbb{R}$, tem-se que:

$$
[\alpha (A + B)]_{ij} 
= \alpha[A + B]_{ij}
= \alpha([A]_{ij} + [B]_{ij}) $$ \ $$
= \alpha [A]_{ij} + \alpha [B]_{ij}
= [\alpha A]_{ij} + [\alpha B]_{ij}
= [\alpha A + \alpha B]_{ij}.
$$ \\


\textbf{$(\alpha \pm \beta) A = \alpha A \pm \beta A$ (Distributiva)}

Considerando que $\alpha$, $\beta$ e todos os elementos $[A]_{ij}$ são números reais, tem-se que:

$$
[(\alpha + \beta) A]_{ij}
= (\alpha + \beta)[A]_{ij}
= \alpha [A]_{ij} + \beta [A]_{ij} 
= [ \alpha A]_{ij} + [\beta A]_{ij}
= [ \alpha A + \beta A]_{ij}.
$$ \\



\item \textbf{Produto de Matrizes}

\textbf{$A + B = B + A$ (Comutativa)}

$A + B = [A]_{ij} + [B]_{ij} = [B]_{ij} + [A]_{ij} = B + A$. \\ \\


\textbf{$(A + B) + C = A + (B + C)$ (Associativa)}

$$
(A + B) + C 
= [A]_{ij} + ([B]_{ij} + [C]_{ij})
= ([A]_{ij} + ([B]_{ij} + [C]_{ij})) $$ \ $$
= (([A]_{ij} + [B]_{ij}) + [C]_{ij})
= ([A]_{ij} + [B]_{ij}) + [C]_{ij}
= A + (B + C).
$$ \\


\textbf{$A(BC) = A(BC)$ (Associativa)}

Sejam A, B e C matrizes m x p, p x q e q x n respectivamente. Tem-se que:

\begin{multicols}{2}
\begin{flushright}
$
[c_{ij}]
$
\columnbreak
\end{flushright}
\begin{flushleft}
$ = \sum\limits_{k=1}^{p} (AB)_{ik} \cdot c_{kj}$ \\
$ = \sum\limits_{k=1}^{p} (\sum\limits_{l=1}^{n} a_{il} \cdot b_{lk}) \cdot c_{kj}$ \\
$ = \sum\limits_{l=1}^{n} a_{il} \cdot (\sum\limits_{k=1}^{p} b_{lk} \cdot c_{kj})$ \\
$ = \sum\limits_{l=1}^{n} a_{il} \cdot (BC)_{lj} = [e_{ij}]$. \\
\end{flushleft}
\end{multicols}

\textbf{$(B + C)A = BA + CA$ (Distributiva)} \\

Usando a definição de produto de matrizes, sejam
$ A = [a_{ij}]_{m \times n}, $ 
$ B = [b_{ij}]_{n \times p}, $ 
$ C = [c_{ij}]_{n \times p}, $ 
$ AB = [e_{ij}]_{m \times p}, $ 
$ AC = [f_{ij}]_{m \times p}, $ 
e $ A(B+C) = [d_{ij}]_{m \times p} $, pode-se dizer que: 

\begin{multicols}{2}
\begin{flushright}
$
[d_{ij}]
$
\columnbreak
\end{flushright}
\begin{flushleft}
$ = \sum\limits_{k=1}^{n} (a_{ik} + b_{ik}) \cdot c_{kj}$ \\
$ = \sum\limits_{k=1}^{n} (a_{ik} \cdot c_{kj} + b_{ik} \cdot c_{kj})$ \\
$ = \sum\limits_{k=1}^{n} a_{ik} \cdot c_{kj} + \sum\limits_{k=1}^{n} b_{ik} \cdot c_{kj}$ \\
$ = [e_{ij}] + [f_{ij}] $.

\end{flushleft}
\end{multicols}


\end{enumerate}
\end{document}
